%\bibliographystyle{sjtu2}%[此处用于每章都生产参考文献]
\chapter{相关技术背景}
\label{chap:back}

\section{传统索引结构}

\subsection{B树及其变种}

B树及其变种索引结构是最常被使用的索引结构之一\cite{graefe2001b}。

% 在计算机科学中,B树(英语:B-tree)是一种自平衡的树,能够保持数据有序。
% 这种数据结构能够让查找数据、顺序访问、插入数据及删除的动作,都在对数时间内完成。
% B树,概括来说是一个一般化的二叉查找树(binary search tree)一个节点可以拥有最少2个子节点。
% 与自平衡二叉查找树不同,B树适用于读写相对大的数据块的存储系统,例如磁盘。
% B树减少定位记录时所经历的中间过程,从而加快存取速度。
% B树这种数据结构可以用来描述外部存储。
% 这种数据结构常被应用在数据库和文件系统的实现上。

Fractal tree and B$\epsilon$-tree

Masstree

A-tree

Wormhole

% uses a fraction of node storage to serve as an
% update buffer \cite{esmet2012tokufs, bender2015and}.
% The updates will be flush to children's buffer when current node's buffer is full and applied until they
% reach the leaf.
% This optimization aims to avoid frequent small writes to disk.

% However, propagating updates also introduce write amplification problem.
% \masst \cite{mao2012cache} partitions key into 8-byte segments and index them with a trie structure.
% Within each trie node, a concurrent \bt is used to index the segments.
% Similar to \sys, \masst uses optimistic concurrency control to ensure read/write atomicity and uses
% fine-grained locks to protect nodes during split and merge.

\subsection{字典树及其变种}

Hot

SuRF

\subsection{哈希表及其变种}

Cuckoo Hash

Level Hash

\subsection{布隆过滤器}

Bloom filter

\section{机器学习}

\subsection{监督学习}

\subsection{回归问题}

\subsection{自动机器学习}

\section{学习索引结构}

\subsection{递归模型索引}
