%\bibliographystyle{sjtu2}%[此处用于每章都生产参考文献]
\chapter{总结与展望}
\label{chap:concl}

本文对{\li}在真实场景下的性能表现进行了讨论,就访问模式与动态数据分布这两个广泛存在于真实场景的工作负载特性进行了深入讨论,
分析并验证了这些特征对{\li}性能带来的影响以及背后的原因。
针对真实场景对{\li}提出的挑战,本文提出了一个新的针对动态场景的学习索引系统{------}{\sys}。
通过\textbf{数据拉伸}方法{\sys}将访问模式信息加入了{\li}的构建过程,通过\textbf{模型缓存}机制{\sys}克服了高昂的{\li}架构搜索代价,
通过完整的系统设计,{\sys}应用以上两个技术创新点,构建了一个面向真实场景的动态的学习索引系统。
实验结果证明,{\sys}能够带来高达79.9\%的性能提升,能够高效地应对真实场景对{\li}带来的挑战。

% This paper proposes a system which can incorperate the
% query distribution in the training set to improve the query
% performance, and reuse the pre-trained model to reduce the
% re-trained cost. 

{\sys}的设计与实现中仍有部分较为初步需要改进的地方,诸如对数据分布与访问模式改变的检测方式、对数据分布的表示方式等。
同时,{\li}作为一个索引结构,其原理与性质与传统索引结构有较大的不同。
我们期望更多对{\li}本质特性的的分析研究工作。
除此之外,{\li}指出了一个较新的研究方向。
如何高效地使用{\model}完成索引,甚至其他系统中的关键操作,仍然是一个吸引人的研究方向。
