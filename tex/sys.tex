%\bibliographystyle{sjtu2}%[此处用于每章都生产参考文献]
\chapter{针对动态场景的学习索引系统:\sys}
\label{chap:sys}

\section{总览}

% To achieve learned indexes' best performance, we propose
% a new learned index system for dynamic workloads called \sys (\Cref{fig:sys}).
% \sys incorporates read access pattern using the \textbf{Training Set Generator} and the
% \textbf{Finalizer} and reuses pre-trained models using the \textbf{Counselor}.

\section{训练集生成器}

% To incorporate read access pattern, an intuitive solution is to increase the 
% contribution of frequently accessed keys during the training process. 
% This can be achieved by creating multiple copies of those keys in the 
% training set. For example, considering a training set of 
% \{(a, 0), (b, 1), (c, 2)\}, where the first element is 
% the key and the second is its position. If the accessed ratio is 1:2:1,
% then we double \emph{b} in the training set,
% which becomes \{(a, 0), (b, 1), (b, 1), (c, 2)\}. 
% In this way, the model will be trained with (b, 1) two times more than others, the prediction 
% accuracy of \emph{b} can be improved. We evaluate this intuitive 
% solution with the workload of Skewed 3 and the dataset D1. With the new 
% training set, the best architecture we can find is NN16 with 275 ns average search time, 
% which is close to the previous best architecture, 282 ns. This is because the intuitive solution does not 
% improve the error bounds of the second stage models
% which decide the search time. In the above evaluation, the 
% average error bound does not improve much (5.21 vs. 5.31). 


% % 2.2 Our solution: stretching the dataset to let the model 
% % of high frequent access to have less keys. 
% % 	2.2.1 Insight, the maximal error with more keys likely 
% % 	to have small errors. 
% % 	2.2.2 Streching dataset allows the model with freq keys have 
% % 	less keys.
% % 	2.2.3 We use the following formular to do the stectch.
% % 		2.2.3.1 fomular
% % 		2.2.3.2 discription
% %		2.2.3.3 example  
% % 	2.2.5 Figure shows the CDF after stretching
% %---------------------------
% \begin{figure}
%     \begin{center}
%         \includegraphics[width=0.9\columnwidth]{figure/results/stretch.eps}
%     \end{center}
%     \caption{
%         \label{fig:stretch}
%         The left figure shows CDF of original D1, while the right figure shows CDF of D1 after stretched.
%     }
%     \end{figure}
%     %---------------------------
% \textbf{``Stretch'' the dataset.} Instead of improving the prediction 
% accuracy of the hot keys, we should focus on the error bounds 
% of the models containing the hot keys (hot models). Since the models assigned with few keys tend to 
% have small error bounds, we try to reduce the number of keys handled by
% the hot models by ``stretching'' the dataset.
% If a key is frequently accessed, we would like to increase 
% the distance between it with its neighbors, the key before or 
% after it. It can be achieved by simply shifting the position 
% labels. 
% Specifically, given a key with position $p$ before ``stretching'', if its access frequency is $f$, and the dataset size is $N$
% then we need to shift its position to be $p + (n-1)/2$, and shift 
% all keys after it with $n-1$. For the above example, the training 
% set of \{(a, 0), (b, 1), (c, 2)\} with access 
% frequency 1:2:1 will be augmented to be \{(a, 0), (b, 1.5), (c, 3)\}. Figure~\ref{fig:stretch} shows the CDF of dataset 1 
% before and after ``stretching'' with the access pattern in workload 
% Skewed 3. 

% Training Set Generator takes the workload and dataset as 
% input, extracts the access pattern by uniformly sampling 
% from the workload and stretches the dataset according to the access 
% pattern. Then it sends the stretched training set to Counselor 
% to get a tuned model. 

% Before using the returned model from Counselor, the Finalizer 
% needs to retrain the last stage models with the original dataset. 
% This is because the position of each key in the stretched training 
% set is changed, we need to repair the position information 
% with the original dataset. This process is considerably
% fast as last models are usually linear models. For example, 
% it only takes 118 $\mu$s to retrain one last model with 1000 keys.

\subsection{数据拉伸}

\section{咨询器}

% After incorporating the access pattern, the only factor affecting
% the model architecture is data distribution.
% We notice that the best model architecture tends to be the same for similar data distributions.
% As a result, \sys is able to cache a mapping from data distributions to 
% models for future reusing. 

% This is done by the Counselor component, which includes 
% four modules: 

% \textbf{Analyzer:} extracts distribution information 
% by uniformly sampling K records from the generated training set, 
% then normalize both key and position to [0, 1]. However, K needs 
% to be large enough to avoid 
% breaking the distribution. 

% \textbf{Model cache:} maintains a mapping 
% from the distribution of previous training set to their learning 
% model's architure and parameters. If it receives a distribution 
% from Analyzer, 
% it will finds the entry in the map with the most similar 
% distribution 
% based on the mean square error. Then, it will send the model's 
% information in that entry to Fine Tuner. Furthermore, if the 
% similarity 
% is below a threshold, it will also start the auto-tuning process. 

% \textbf{Fine Tuner:} incrementally trains the model retrieved from 
% the model cache with the training set. 

% \textbf{Auto-tuner:} uses grid search to find the best model architecture in the given search 
% space. 
% It performs auto-tuning at the background and sends the result to 
% the Finalizer component.

\subsection{数据集匹配}

\section{终结器}

\section{讨论}

% \textbf{Detecting the change of distribution and access pattern.} 
% {\sys} will start to run on detecting the change of distribution or 
% access pattern. The detection must be timely with few false positive.
% For currently design, we simply detect this by monitoring the 
% degradation of the peformance. However, we can use similar technique 
% in~\cite{kang2017noscope} to improve the accuracy. 

% \textbf{Extract the distribution feature from a dataset.} Currently, we 
% simply extract the distribution by uniformly sampling the dataset. 
% However, to avoid breaking the distribution, the sample rate varies 
% across different dataset. As a result, it is challengin the decide the 
% sample rate.

% \textbf{Compute the similarity.}
% Our sampled distribution representation can be regarded as a type of sequential
% data, for which there are many machine learning models are targeting
% \cite{greff2017lstm, cho2014learning}.
% We believe we can further leverage learning to learn a better similarity metric.

% \textbf{Efficiently find the similar distribution in Model Cache.} There can be 
% throusands to millions entires in the Model Cache. As a result, 
% finding the entry with most similar distribution is considerably 
% cost. To solve this issue, we plan to use methods like
% \cite{metwally2005efficient, ilyas2008survey} to first filter out the most relevant
% entries before the comparison.

% \textbf{Improve Auto-tuner efficiency.}
% Grid search is slow.
% To speed up the search, there are works that use Gaussian process to optimize
% the search process \cite{snoek2012practical, bergstra2011algorithms, brochu2010tutorial}.
% Similar ideas are also used in database system optimization \cite{duan2009tuning, thummala2010ituned}.